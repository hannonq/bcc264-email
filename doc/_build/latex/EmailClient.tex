% Generated by Sphinx.
\def\sphinxdocclass{report}
\documentclass[letterpaper,10pt,english]{sphinxmanual}
\usepackage[utf8]{inputenc}
\DeclareUnicodeCharacter{00A0}{\nobreakspace}
\usepackage{cmap}
\usepackage[T1]{fontenc}
\usepackage{babel}
\usepackage{times}
\usepackage[Sonny]{fncychap}
\usepackage{longtable}
\usepackage{sphinx}
\usepackage{multirow}
\usepackage{eqparbox}


\addto\captionsenglish{\renewcommand{\figurename}{Fig. }}
\addto\captionsenglish{\renewcommand{\tablename}{Table }}
\SetupFloatingEnvironment{literal-block}{name=Listing }



\title{Email Client Documentation}
\date{January 27, 2016}
\release{1.0}
\author{Hannon, Marcos Vinícius}
\newcommand{\sphinxlogo}{}
\renewcommand{\releasename}{Release}
\setcounter{tocdepth}{1}
\makeindex

\makeatletter
\def\PYG@reset{\let\PYG@it=\relax \let\PYG@bf=\relax%
    \let\PYG@ul=\relax \let\PYG@tc=\relax%
    \let\PYG@bc=\relax \let\PYG@ff=\relax}
\def\PYG@tok#1{\csname PYG@tok@#1\endcsname}
\def\PYG@toks#1+{\ifx\relax#1\empty\else%
    \PYG@tok{#1}\expandafter\PYG@toks\fi}
\def\PYG@do#1{\PYG@bc{\PYG@tc{\PYG@ul{%
    \PYG@it{\PYG@bf{\PYG@ff{#1}}}}}}}
\def\PYG#1#2{\PYG@reset\PYG@toks#1+\relax+\PYG@do{#2}}

\expandafter\def\csname PYG@tok@sb\endcsname{\def\PYG@tc##1{\textcolor[rgb]{0.25,0.44,0.63}{##1}}}
\expandafter\def\csname PYG@tok@nb\endcsname{\def\PYG@tc##1{\textcolor[rgb]{0.00,0.44,0.13}{##1}}}
\expandafter\def\csname PYG@tok@ch\endcsname{\let\PYG@it=\textit\def\PYG@tc##1{\textcolor[rgb]{0.25,0.50,0.56}{##1}}}
\expandafter\def\csname PYG@tok@sc\endcsname{\def\PYG@tc##1{\textcolor[rgb]{0.25,0.44,0.63}{##1}}}
\expandafter\def\csname PYG@tok@gp\endcsname{\let\PYG@bf=\textbf\def\PYG@tc##1{\textcolor[rgb]{0.78,0.36,0.04}{##1}}}
\expandafter\def\csname PYG@tok@nf\endcsname{\def\PYG@tc##1{\textcolor[rgb]{0.02,0.16,0.49}{##1}}}
\expandafter\def\csname PYG@tok@ow\endcsname{\let\PYG@bf=\textbf\def\PYG@tc##1{\textcolor[rgb]{0.00,0.44,0.13}{##1}}}
\expandafter\def\csname PYG@tok@nt\endcsname{\let\PYG@bf=\textbf\def\PYG@tc##1{\textcolor[rgb]{0.02,0.16,0.45}{##1}}}
\expandafter\def\csname PYG@tok@c\endcsname{\let\PYG@it=\textit\def\PYG@tc##1{\textcolor[rgb]{0.25,0.50,0.56}{##1}}}
\expandafter\def\csname PYG@tok@w\endcsname{\def\PYG@tc##1{\textcolor[rgb]{0.73,0.73,0.73}{##1}}}
\expandafter\def\csname PYG@tok@k\endcsname{\let\PYG@bf=\textbf\def\PYG@tc##1{\textcolor[rgb]{0.00,0.44,0.13}{##1}}}
\expandafter\def\csname PYG@tok@kt\endcsname{\def\PYG@tc##1{\textcolor[rgb]{0.56,0.13,0.00}{##1}}}
\expandafter\def\csname PYG@tok@s1\endcsname{\def\PYG@tc##1{\textcolor[rgb]{0.25,0.44,0.63}{##1}}}
\expandafter\def\csname PYG@tok@ge\endcsname{\let\PYG@it=\textit}
\expandafter\def\csname PYG@tok@no\endcsname{\def\PYG@tc##1{\textcolor[rgb]{0.38,0.68,0.84}{##1}}}
\expandafter\def\csname PYG@tok@cs\endcsname{\def\PYG@tc##1{\textcolor[rgb]{0.25,0.50,0.56}{##1}}\def\PYG@bc##1{\setlength{\fboxsep}{0pt}\colorbox[rgb]{1.00,0.94,0.94}{\strut ##1}}}
\expandafter\def\csname PYG@tok@gt\endcsname{\def\PYG@tc##1{\textcolor[rgb]{0.00,0.27,0.87}{##1}}}
\expandafter\def\csname PYG@tok@gi\endcsname{\def\PYG@tc##1{\textcolor[rgb]{0.00,0.63,0.00}{##1}}}
\expandafter\def\csname PYG@tok@se\endcsname{\let\PYG@bf=\textbf\def\PYG@tc##1{\textcolor[rgb]{0.25,0.44,0.63}{##1}}}
\expandafter\def\csname PYG@tok@kd\endcsname{\let\PYG@bf=\textbf\def\PYG@tc##1{\textcolor[rgb]{0.00,0.44,0.13}{##1}}}
\expandafter\def\csname PYG@tok@si\endcsname{\let\PYG@it=\textit\def\PYG@tc##1{\textcolor[rgb]{0.44,0.63,0.82}{##1}}}
\expandafter\def\csname PYG@tok@vc\endcsname{\def\PYG@tc##1{\textcolor[rgb]{0.73,0.38,0.84}{##1}}}
\expandafter\def\csname PYG@tok@sd\endcsname{\let\PYG@it=\textit\def\PYG@tc##1{\textcolor[rgb]{0.25,0.44,0.63}{##1}}}
\expandafter\def\csname PYG@tok@kr\endcsname{\let\PYG@bf=\textbf\def\PYG@tc##1{\textcolor[rgb]{0.00,0.44,0.13}{##1}}}
\expandafter\def\csname PYG@tok@kc\endcsname{\let\PYG@bf=\textbf\def\PYG@tc##1{\textcolor[rgb]{0.00,0.44,0.13}{##1}}}
\expandafter\def\csname PYG@tok@bp\endcsname{\def\PYG@tc##1{\textcolor[rgb]{0.00,0.44,0.13}{##1}}}
\expandafter\def\csname PYG@tok@gs\endcsname{\let\PYG@bf=\textbf}
\expandafter\def\csname PYG@tok@vg\endcsname{\def\PYG@tc##1{\textcolor[rgb]{0.73,0.38,0.84}{##1}}}
\expandafter\def\csname PYG@tok@sh\endcsname{\def\PYG@tc##1{\textcolor[rgb]{0.25,0.44,0.63}{##1}}}
\expandafter\def\csname PYG@tok@s2\endcsname{\def\PYG@tc##1{\textcolor[rgb]{0.25,0.44,0.63}{##1}}}
\expandafter\def\csname PYG@tok@mf\endcsname{\def\PYG@tc##1{\textcolor[rgb]{0.13,0.50,0.31}{##1}}}
\expandafter\def\csname PYG@tok@mb\endcsname{\def\PYG@tc##1{\textcolor[rgb]{0.13,0.50,0.31}{##1}}}
\expandafter\def\csname PYG@tok@gh\endcsname{\let\PYG@bf=\textbf\def\PYG@tc##1{\textcolor[rgb]{0.00,0.00,0.50}{##1}}}
\expandafter\def\csname PYG@tok@o\endcsname{\def\PYG@tc##1{\textcolor[rgb]{0.40,0.40,0.40}{##1}}}
\expandafter\def\csname PYG@tok@vi\endcsname{\def\PYG@tc##1{\textcolor[rgb]{0.73,0.38,0.84}{##1}}}
\expandafter\def\csname PYG@tok@sx\endcsname{\def\PYG@tc##1{\textcolor[rgb]{0.78,0.36,0.04}{##1}}}
\expandafter\def\csname PYG@tok@c1\endcsname{\let\PYG@it=\textit\def\PYG@tc##1{\textcolor[rgb]{0.25,0.50,0.56}{##1}}}
\expandafter\def\csname PYG@tok@go\endcsname{\def\PYG@tc##1{\textcolor[rgb]{0.20,0.20,0.20}{##1}}}
\expandafter\def\csname PYG@tok@mh\endcsname{\def\PYG@tc##1{\textcolor[rgb]{0.13,0.50,0.31}{##1}}}
\expandafter\def\csname PYG@tok@ne\endcsname{\def\PYG@tc##1{\textcolor[rgb]{0.00,0.44,0.13}{##1}}}
\expandafter\def\csname PYG@tok@na\endcsname{\def\PYG@tc##1{\textcolor[rgb]{0.25,0.44,0.63}{##1}}}
\expandafter\def\csname PYG@tok@ss\endcsname{\def\PYG@tc##1{\textcolor[rgb]{0.32,0.47,0.09}{##1}}}
\expandafter\def\csname PYG@tok@nc\endcsname{\let\PYG@bf=\textbf\def\PYG@tc##1{\textcolor[rgb]{0.05,0.52,0.71}{##1}}}
\expandafter\def\csname PYG@tok@gd\endcsname{\def\PYG@tc##1{\textcolor[rgb]{0.63,0.00,0.00}{##1}}}
\expandafter\def\csname PYG@tok@gu\endcsname{\let\PYG@bf=\textbf\def\PYG@tc##1{\textcolor[rgb]{0.50,0.00,0.50}{##1}}}
\expandafter\def\csname PYG@tok@ni\endcsname{\let\PYG@bf=\textbf\def\PYG@tc##1{\textcolor[rgb]{0.84,0.33,0.22}{##1}}}
\expandafter\def\csname PYG@tok@err\endcsname{\def\PYG@bc##1{\setlength{\fboxsep}{0pt}\fcolorbox[rgb]{1.00,0.00,0.00}{1,1,1}{\strut ##1}}}
\expandafter\def\csname PYG@tok@s\endcsname{\def\PYG@tc##1{\textcolor[rgb]{0.25,0.44,0.63}{##1}}}
\expandafter\def\csname PYG@tok@kp\endcsname{\def\PYG@tc##1{\textcolor[rgb]{0.00,0.44,0.13}{##1}}}
\expandafter\def\csname PYG@tok@cm\endcsname{\let\PYG@it=\textit\def\PYG@tc##1{\textcolor[rgb]{0.25,0.50,0.56}{##1}}}
\expandafter\def\csname PYG@tok@cpf\endcsname{\let\PYG@it=\textit\def\PYG@tc##1{\textcolor[rgb]{0.25,0.50,0.56}{##1}}}
\expandafter\def\csname PYG@tok@m\endcsname{\def\PYG@tc##1{\textcolor[rgb]{0.13,0.50,0.31}{##1}}}
\expandafter\def\csname PYG@tok@nv\endcsname{\def\PYG@tc##1{\textcolor[rgb]{0.73,0.38,0.84}{##1}}}
\expandafter\def\csname PYG@tok@cp\endcsname{\def\PYG@tc##1{\textcolor[rgb]{0.00,0.44,0.13}{##1}}}
\expandafter\def\csname PYG@tok@mi\endcsname{\def\PYG@tc##1{\textcolor[rgb]{0.13,0.50,0.31}{##1}}}
\expandafter\def\csname PYG@tok@gr\endcsname{\def\PYG@tc##1{\textcolor[rgb]{1.00,0.00,0.00}{##1}}}
\expandafter\def\csname PYG@tok@nl\endcsname{\let\PYG@bf=\textbf\def\PYG@tc##1{\textcolor[rgb]{0.00,0.13,0.44}{##1}}}
\expandafter\def\csname PYG@tok@il\endcsname{\def\PYG@tc##1{\textcolor[rgb]{0.13,0.50,0.31}{##1}}}
\expandafter\def\csname PYG@tok@sr\endcsname{\def\PYG@tc##1{\textcolor[rgb]{0.14,0.33,0.53}{##1}}}
\expandafter\def\csname PYG@tok@mo\endcsname{\def\PYG@tc##1{\textcolor[rgb]{0.13,0.50,0.31}{##1}}}
\expandafter\def\csname PYG@tok@kn\endcsname{\let\PYG@bf=\textbf\def\PYG@tc##1{\textcolor[rgb]{0.00,0.44,0.13}{##1}}}
\expandafter\def\csname PYG@tok@nn\endcsname{\let\PYG@bf=\textbf\def\PYG@tc##1{\textcolor[rgb]{0.05,0.52,0.71}{##1}}}
\expandafter\def\csname PYG@tok@nd\endcsname{\let\PYG@bf=\textbf\def\PYG@tc##1{\textcolor[rgb]{0.33,0.33,0.33}{##1}}}

\def\PYGZbs{\char`\\}
\def\PYGZus{\char`\_}
\def\PYGZob{\char`\{}
\def\PYGZcb{\char`\}}
\def\PYGZca{\char`\^}
\def\PYGZam{\char`\&}
\def\PYGZlt{\char`\<}
\def\PYGZgt{\char`\>}
\def\PYGZsh{\char`\#}
\def\PYGZpc{\char`\%}
\def\PYGZdl{\char`\$}
\def\PYGZhy{\char`\-}
\def\PYGZsq{\char`\'}
\def\PYGZdq{\char`\"}
\def\PYGZti{\char`\~}
% for compatibility with earlier versions
\def\PYGZat{@}
\def\PYGZlb{[}
\def\PYGZrb{]}
\makeatother

\renewcommand\PYGZsq{\textquotesingle}

\begin{document}

\maketitle
\tableofcontents
\phantomsection\label{index::doc}


Contents:


\chapter{push\_email package}
\label{push_email:push-email-package}\label{push_email:welcome-to-email-client-s-documentation}\label{push_email::doc}

\section{Subpackages}
\label{push_email:subpackages}

\subsection{push\_email.migrations package}
\label{push_email.migrations:push-email-migrations-package}\label{push_email.migrations::doc}

\subsubsection{Submodules}
\label{push_email.migrations:submodules}

\subsubsection{push\_email.migrations.0001\_initial module}
\label{push_email.migrations:module-push_email.migrations.0001_initial}\label{push_email.migrations:push-email-migrations-0001-initial-module}\index{push\_email.migrations.0001\_initial (module)}\index{Migration (class in push\_email.migrations.0001\_initial)}

\begin{fulllineitems}
\phantomsection\label{push_email.migrations:push_email.migrations.0001_initial.Migration}\pysiglinewithargsret{\strong{class }\code{push\_email.migrations.0001\_initial.}\bfcode{Migration}}{\emph{name}, \emph{app\_label}}{}
Bases: \code{django.db.migrations.migration.Migration}
\index{dependencies (push\_email.migrations.0001\_initial.Migration attribute)}

\begin{fulllineitems}
\phantomsection\label{push_email.migrations:push_email.migrations.0001_initial.Migration.dependencies}\pysigline{\bfcode{dependencies}\strong{ = {[}{]}}}
\end{fulllineitems}

\index{initial (push\_email.migrations.0001\_initial.Migration attribute)}

\begin{fulllineitems}
\phantomsection\label{push_email.migrations:push_email.migrations.0001_initial.Migration.initial}\pysigline{\bfcode{initial}\strong{ = True}}
\end{fulllineitems}

\index{operations (push\_email.migrations.0001\_initial.Migration attribute)}

\begin{fulllineitems}
\phantomsection\label{push_email.migrations:push_email.migrations.0001_initial.Migration.operations}\pysigline{\bfcode{operations}\strong{ = {[}\textless{}CreateModel  fields={[}(`id', \textless{}django.db.models.fields.AutoField\textgreater{}), (`copy\_to', \textless{}django.db.models.fields.EmailField\textgreater{}){]}, name='Copy'\textgreater{}, \textless{}CreateModel  fields={[}(`id', \textless{}django.db.models.fields.AutoField\textgreater{}), (`email\_subject', \textless{}django.db.models.fields.CharField\textgreater{}), (`email\_from', \textless{}django.db.models.fields.EmailField\textgreater{}), (`email\_body', \textless{}django.db.models.fields.TextField\textgreater{}), (`date', \textless{}django.db.models.fields.DateTimeField\textgreater{}), (`copies', \textless{}django.db.models.fields.related.ManyToManyField\textgreater{}){]}, name='MyEmail'\textgreater{}, \textless{}CreateModel  fields={[}(`id', \textless{}django.db.models.fields.AutoField\textgreater{}), (`email\_to', \textless{}django.db.models.fields.EmailField\textgreater{}){]}, name='Recipient'\textgreater{}, \textless{}AddField  name='recipients', model\_name='myemail', field=\textless{}django.db.models.fields.related.ManyToManyField\textgreater{}\textgreater{}{]}}}
\end{fulllineitems}


\end{fulllineitems}



\subsubsection{push\_email.migrations.0002\_auto\_20160111\_2042 module}
\label{push_email.migrations:module-push_email.migrations.0002_auto_20160111_2042}\label{push_email.migrations:push-email-migrations-0002-auto-20160111-2042-module}\index{push\_email.migrations.0002\_auto\_20160111\_2042 (module)}\index{Migration (class in push\_email.migrations.0002\_auto\_20160111\_2042)}

\begin{fulllineitems}
\phantomsection\label{push_email.migrations:push_email.migrations.0002_auto_20160111_2042.Migration}\pysiglinewithargsret{\strong{class }\code{push\_email.migrations.0002\_auto\_20160111\_2042.}\bfcode{Migration}}{\emph{name}, \emph{app\_label}}{}
Bases: \code{django.db.migrations.migration.Migration}
\index{dependencies (push\_email.migrations.0002\_auto\_20160111\_2042.Migration attribute)}

\begin{fulllineitems}
\phantomsection\label{push_email.migrations:push_email.migrations.0002_auto_20160111_2042.Migration.dependencies}\pysigline{\bfcode{dependencies}\strong{ = {[}(`push\_email', `0001\_initial'){]}}}
\end{fulllineitems}

\index{operations (push\_email.migrations.0002\_auto\_20160111\_2042.Migration attribute)}

\begin{fulllineitems}
\phantomsection\label{push_email.migrations:push_email.migrations.0002_auto_20160111_2042.Migration.operations}\pysigline{\bfcode{operations}\strong{ = {[}\textless{}RenameField  new\_name='email\_address', old\_name='copy\_to', model\_name='copy'\textgreater{}, \textless{}RenameField  new\_name='email\_address', old\_name='email\_to', model\_name='recipient'\textgreater{}, \textless{}AlterField  name='copies', model\_name='myemail', field=\textless{}django.db.models.fields.related.ManyToManyField\textgreater{}\textgreater{}{]}}}
\end{fulllineitems}


\end{fulllineitems}



\subsubsection{push\_email.migrations.0003\_auto\_20160111\_2043 module}
\label{push_email.migrations:module-push_email.migrations.0003_auto_20160111_2043}\label{push_email.migrations:push-email-migrations-0003-auto-20160111-2043-module}\index{push\_email.migrations.0003\_auto\_20160111\_2043 (module)}\index{Migration (class in push\_email.migrations.0003\_auto\_20160111\_2043)}

\begin{fulllineitems}
\phantomsection\label{push_email.migrations:push_email.migrations.0003_auto_20160111_2043.Migration}\pysiglinewithargsret{\strong{class }\code{push\_email.migrations.0003\_auto\_20160111\_2043.}\bfcode{Migration}}{\emph{name}, \emph{app\_label}}{}
Bases: \code{django.db.migrations.migration.Migration}
\index{dependencies (push\_email.migrations.0003\_auto\_20160111\_2043.Migration attribute)}

\begin{fulllineitems}
\phantomsection\label{push_email.migrations:push_email.migrations.0003_auto_20160111_2043.Migration.dependencies}\pysigline{\bfcode{dependencies}\strong{ = {[}(`push\_email', `0002\_auto\_20160111\_2042'){]}}}
\end{fulllineitems}

\index{operations (push\_email.migrations.0003\_auto\_20160111\_2043.Migration attribute)}

\begin{fulllineitems}
\phantomsection\label{push_email.migrations:push_email.migrations.0003_auto_20160111_2043.Migration.operations}\pysigline{\bfcode{operations}\strong{ = {[}\textless{}AlterField  name='copies', model\_name='myemail', field=\textless{}django.db.models.fields.related.ManyToManyField\textgreater{}\textgreater{}{]}}}
\end{fulllineitems}


\end{fulllineitems}



\subsubsection{push\_email.migrations.0004\_emailid module}
\label{push_email.migrations:module-push_email.migrations.0004_emailid}\label{push_email.migrations:push-email-migrations-0004-emailid-module}\index{push\_email.migrations.0004\_emailid (module)}\index{Migration (class in push\_email.migrations.0004\_emailid)}

\begin{fulllineitems}
\phantomsection\label{push_email.migrations:push_email.migrations.0004_emailid.Migration}\pysiglinewithargsret{\strong{class }\code{push\_email.migrations.0004\_emailid.}\bfcode{Migration}}{\emph{name}, \emph{app\_label}}{}
Bases: \code{django.db.migrations.migration.Migration}
\index{dependencies (push\_email.migrations.0004\_emailid.Migration attribute)}

\begin{fulllineitems}
\phantomsection\label{push_email.migrations:push_email.migrations.0004_emailid.Migration.dependencies}\pysigline{\bfcode{dependencies}\strong{ = {[}(`push\_email', `0003\_auto\_20160111\_2043'){]}}}
\end{fulllineitems}

\index{operations (push\_email.migrations.0004\_emailid.Migration attribute)}

\begin{fulllineitems}
\phantomsection\label{push_email.migrations:push_email.migrations.0004_emailid.Migration.operations}\pysigline{\bfcode{operations}\strong{ = {[}\textless{}CreateModel  fields={[}(`id', \textless{}django.db.models.fields.AutoField\textgreater{}), (`email\_id', \textless{}django.db.models.fields.TextField\textgreater{}){]}, name='EmailId'\textgreater{}{]}}}
\end{fulllineitems}


\end{fulllineitems}



\subsubsection{Module contents}
\label{push_email.migrations:module-push_email.migrations}\label{push_email.migrations:module-contents}\index{push\_email.migrations (module)}

\section{Submodules}
\label{push_email:submodules}

\section{push\_email.admin module}
\label{push_email:push-email-admin-module}\label{push_email:module-push_email.admin}\index{push\_email.admin (module)}

\section{push\_email.apps module}
\label{push_email:module-push_email.apps}\label{push_email:push-email-apps-module}\index{push\_email.apps (module)}\index{PushEmailConfig (class in push\_email.apps)}

\begin{fulllineitems}
\phantomsection\label{push_email:push_email.apps.PushEmailConfig}\pysiglinewithargsret{\strong{class }\code{push\_email.apps.}\bfcode{PushEmailConfig}}{\emph{app\_name}, \emph{app\_module}}{}
Bases: \code{django.apps.config.AppConfig}
\index{name (push\_email.apps.PushEmailConfig attribute)}

\begin{fulllineitems}
\phantomsection\label{push_email:push_email.apps.PushEmailConfig.name}\pysigline{\bfcode{name}\strong{ = `push\_email'}}
\end{fulllineitems}


\end{fulllineitems}



\section{push\_email.calendarhandler module}
\label{push_email:push-email-calendarhandler-module}

\section{push\_email.forms module}
\label{push_email:push-email-forms-module}\label{push_email:module-push_email.forms}\index{push\_email.forms (module)}\index{LoginForm (class in push\_email.forms)}

\begin{fulllineitems}
\phantomsection\label{push_email:push_email.forms.LoginForm}\pysiglinewithargsret{\strong{class }\code{push\_email.forms.}\bfcode{LoginForm}}{\emph{data=None}, \emph{files=None}, \emph{auto\_id='id\_\%s'}, \emph{prefix=None}, \emph{initial=None}, \emph{error\_class=\textless{}class `django.forms.utils.ErrorList'\textgreater{}}, \emph{label\_suffix=None}, \emph{empty\_permitted=False}, \emph{field\_order=None}}{}
Bases: \code{django.forms.forms.Form}
\index{base\_fields (push\_email.forms.LoginForm attribute)}

\begin{fulllineitems}
\phantomsection\label{push_email:push_email.forms.LoginForm.base_fields}\pysigline{\bfcode{base\_fields}\strong{ = OrderedDict({[}(`email\_address1', \textless{}django.forms.fields.CharField object at 0x2afe6e6fc518\textgreater{}), (`password1', \textless{}django.forms.fields.CharField object at 0x2afe6e6fc828\textgreater{}), (`email\_address2', \textless{}django.forms.fields.CharField object at 0x2afe6e6fc898\textgreater{}), (`password2', \textless{}django.forms.fields.CharField object at 0x2afe6e6fc908\textgreater{}){]})}}
\end{fulllineitems}

\index{declared\_fields (push\_email.forms.LoginForm attribute)}

\begin{fulllineitems}
\phantomsection\label{push_email:push_email.forms.LoginForm.declared_fields}\pysigline{\bfcode{declared\_fields}\strong{ = OrderedDict({[}(`email\_address1', \textless{}django.forms.fields.CharField object at 0x2afe6e6fc518\textgreater{}), (`password1', \textless{}django.forms.fields.CharField object at 0x2afe6e6fc828\textgreater{}), (`email\_address2', \textless{}django.forms.fields.CharField object at 0x2afe6e6fc898\textgreater{}), (`password2', \textless{}django.forms.fields.CharField object at 0x2afe6e6fc908\textgreater{}){]})}}
\end{fulllineitems}

\index{media (push\_email.forms.LoginForm attribute)}

\begin{fulllineitems}
\phantomsection\label{push_email:push_email.forms.LoginForm.media}\pysigline{\bfcode{media}}
\end{fulllineitems}


\end{fulllineitems}



\section{push\_email.models module}
\label{push_email:push-email-models-module}\label{push_email:module-push_email.models}\index{push\_email.models (module)}\index{Copy (class in push\_email.models)}

\begin{fulllineitems}
\phantomsection\label{push_email:push_email.models.Copy}\pysiglinewithargsret{\strong{class }\code{push\_email.models.}\bfcode{Copy}}{\emph{*args}, \emph{**kwargs}}{}
Bases: \code{django.db.models.base.Model}

Creates a ``copy'' table
\index{Copy.DoesNotExist}

\begin{fulllineitems}
\phantomsection\label{push_email:push_email.models.Copy.DoesNotExist}\pysigline{\strong{exception }\bfcode{DoesNotExist}}
Bases: \code{django.core.exceptions.ObjectDoesNotExist}

\end{fulllineitems}

\index{Copy.MultipleObjectsReturned}

\begin{fulllineitems}
\phantomsection\label{push_email:push_email.models.Copy.MultipleObjectsReturned}\pysigline{\strong{exception }\code{Copy.}\bfcode{MultipleObjectsReturned}}
Bases: \code{django.core.exceptions.MultipleObjectsReturned}

\end{fulllineitems}

\index{myemail\_set (push\_email.models.Copy attribute)}

\begin{fulllineitems}
\phantomsection\label{push_email:push_email.models.Copy.myemail_set}\pysigline{\code{Copy.}\bfcode{myemail\_set}}
Accessor to the related objects manager on the forward and reverse sides of
a many-to-many relation.

In the example:

\begin{Verbatim}[commandchars=\\\{\}]
\PYG{k}{class} \PYG{n+nc}{Pizza}\PYG{p}{(}\PYG{n}{Model}\PYG{p}{)}\PYG{p}{:}
    \PYG{n}{toppings} \PYG{o}{=} \PYG{n}{ManyToManyField}\PYG{p}{(}\PYG{n}{Topping}\PYG{p}{,} \PYG{n}{related\PYGZus{}name}\PYG{o}{=}\PYG{l+s+s1}{\PYGZsq{}}\PYG{l+s+s1}{pizzas}\PYG{l+s+s1}{\PYGZsq{}}\PYG{p}{)}
\end{Verbatim}

\code{pizza.toppings} and \code{topping.pizzas} are \code{ManyToManyDescriptor}
instances.

Most of the implementation is delegated to a dynamically defined manager
class built by \code{create\_forward\_many\_to\_many\_manager()} defined below.

\end{fulllineitems}

\index{objects (push\_email.models.Copy attribute)}

\begin{fulllineitems}
\phantomsection\label{push_email:push_email.models.Copy.objects}\pysigline{\code{Copy.}\bfcode{objects}\strong{ = \textless{}django.db.models.manager.Manager object\textgreater{}}}
\end{fulllineitems}


\end{fulllineitems}

\index{EmailId (class in push\_email.models)}

\begin{fulllineitems}
\phantomsection\label{push_email:push_email.models.EmailId}\pysiglinewithargsret{\strong{class }\code{push\_email.models.}\bfcode{EmailId}}{\emph{*args}, \emph{**kwargs}}{}
Bases: \code{django.db.models.base.Model}

A unique email ID
\index{EmailId.DoesNotExist}

\begin{fulllineitems}
\phantomsection\label{push_email:push_email.models.EmailId.DoesNotExist}\pysigline{\strong{exception }\bfcode{DoesNotExist}}
Bases: \code{django.core.exceptions.ObjectDoesNotExist}

\end{fulllineitems}

\index{EmailId.MultipleObjectsReturned}

\begin{fulllineitems}
\phantomsection\label{push_email:push_email.models.EmailId.MultipleObjectsReturned}\pysigline{\strong{exception }\code{EmailId.}\bfcode{MultipleObjectsReturned}}
Bases: \code{django.core.exceptions.MultipleObjectsReturned}

\end{fulllineitems}

\index{objects (push\_email.models.EmailId attribute)}

\begin{fulllineitems}
\phantomsection\label{push_email:push_email.models.EmailId.objects}\pysigline{\code{EmailId.}\bfcode{objects}\strong{ = \textless{}django.db.models.manager.Manager object\textgreater{}}}
\end{fulllineitems}


\end{fulllineitems}

\index{MyEmail (class in push\_email.models)}

\begin{fulllineitems}
\phantomsection\label{push_email:push_email.models.MyEmail}\pysiglinewithargsret{\strong{class }\code{push\_email.models.}\bfcode{MyEmail}}{\emph{*args}, \emph{**kwargs}}{}
Bases: \code{django.db.models.base.Model}

Email object to save all useful email information
\index{MyEmail.DoesNotExist}

\begin{fulllineitems}
\phantomsection\label{push_email:push_email.models.MyEmail.DoesNotExist}\pysigline{\strong{exception }\bfcode{DoesNotExist}}
Bases: \code{django.core.exceptions.ObjectDoesNotExist}

\end{fulllineitems}

\index{MyEmail.MultipleObjectsReturned}

\begin{fulllineitems}
\phantomsection\label{push_email:push_email.models.MyEmail.MultipleObjectsReturned}\pysigline{\strong{exception }\code{MyEmail.}\bfcode{MultipleObjectsReturned}}
Bases: \code{django.core.exceptions.MultipleObjectsReturned}

\end{fulllineitems}

\index{copies (push\_email.models.MyEmail attribute)}

\begin{fulllineitems}
\phantomsection\label{push_email:push_email.models.MyEmail.copies}\pysigline{\code{MyEmail.}\bfcode{copies}}
Accessor to the related objects manager on the forward and reverse sides of
a many-to-many relation.

In the example:

\begin{Verbatim}[commandchars=\\\{\}]
\PYG{k}{class} \PYG{n+nc}{Pizza}\PYG{p}{(}\PYG{n}{Model}\PYG{p}{)}\PYG{p}{:}
    \PYG{n}{toppings} \PYG{o}{=} \PYG{n}{ManyToManyField}\PYG{p}{(}\PYG{n}{Topping}\PYG{p}{,} \PYG{n}{related\PYGZus{}name}\PYG{o}{=}\PYG{l+s+s1}{\PYGZsq{}}\PYG{l+s+s1}{pizzas}\PYG{l+s+s1}{\PYGZsq{}}\PYG{p}{)}
\end{Verbatim}

\code{pizza.toppings} and \code{topping.pizzas} are \code{ManyToManyDescriptor}
instances.

Most of the implementation is delegated to a dynamically defined manager
class built by \code{create\_forward\_many\_to\_many\_manager()} defined below.

\end{fulllineitems}

\index{get\_next\_by\_date() (push\_email.models.MyEmail method)}

\begin{fulllineitems}
\phantomsection\label{push_email:push_email.models.MyEmail.get_next_by_date}\pysiglinewithargsret{\code{MyEmail.}\bfcode{get\_next\_by\_date}}{\emph{*moreargs}, \emph{**morekwargs}}{}
\end{fulllineitems}

\index{get\_previous\_by\_date() (push\_email.models.MyEmail method)}

\begin{fulllineitems}
\phantomsection\label{push_email:push_email.models.MyEmail.get_previous_by_date}\pysiglinewithargsret{\code{MyEmail.}\bfcode{get\_previous\_by\_date}}{\emph{*moreargs}, \emph{**morekwargs}}{}
\end{fulllineitems}

\index{objects (push\_email.models.MyEmail attribute)}

\begin{fulllineitems}
\phantomsection\label{push_email:push_email.models.MyEmail.objects}\pysigline{\code{MyEmail.}\bfcode{objects}\strong{ = \textless{}django.db.models.manager.Manager object\textgreater{}}}
\end{fulllineitems}

\index{recipients (push\_email.models.MyEmail attribute)}

\begin{fulllineitems}
\phantomsection\label{push_email:push_email.models.MyEmail.recipients}\pysigline{\code{MyEmail.}\bfcode{recipients}}
Accessor to the related objects manager on the forward and reverse sides of
a many-to-many relation.

In the example:

\begin{Verbatim}[commandchars=\\\{\}]
\PYG{k}{class} \PYG{n+nc}{Pizza}\PYG{p}{(}\PYG{n}{Model}\PYG{p}{)}\PYG{p}{:}
    \PYG{n}{toppings} \PYG{o}{=} \PYG{n}{ManyToManyField}\PYG{p}{(}\PYG{n}{Topping}\PYG{p}{,} \PYG{n}{related\PYGZus{}name}\PYG{o}{=}\PYG{l+s+s1}{\PYGZsq{}}\PYG{l+s+s1}{pizzas}\PYG{l+s+s1}{\PYGZsq{}}\PYG{p}{)}
\end{Verbatim}

\code{pizza.toppings} and \code{topping.pizzas} are \code{ManyToManyDescriptor}
instances.

Most of the implementation is delegated to a dynamically defined manager
class built by \code{create\_forward\_many\_to\_many\_manager()} defined below.

\end{fulllineitems}


\end{fulllineitems}

\index{Recipient (class in push\_email.models)}

\begin{fulllineitems}
\phantomsection\label{push_email:push_email.models.Recipient}\pysiglinewithargsret{\strong{class }\code{push\_email.models.}\bfcode{Recipient}}{\emph{*args}, \emph{**kwargs}}{}
Bases: \code{django.db.models.base.Model}

Creates a ``recipient'' table
\index{Recipient.DoesNotExist}

\begin{fulllineitems}
\phantomsection\label{push_email:push_email.models.Recipient.DoesNotExist}\pysigline{\strong{exception }\bfcode{DoesNotExist}}
Bases: \code{django.core.exceptions.ObjectDoesNotExist}

\end{fulllineitems}

\index{Recipient.MultipleObjectsReturned}

\begin{fulllineitems}
\phantomsection\label{push_email:push_email.models.Recipient.MultipleObjectsReturned}\pysigline{\strong{exception }\code{Recipient.}\bfcode{MultipleObjectsReturned}}
Bases: \code{django.core.exceptions.MultipleObjectsReturned}

\end{fulllineitems}

\index{myemail\_set (push\_email.models.Recipient attribute)}

\begin{fulllineitems}
\phantomsection\label{push_email:push_email.models.Recipient.myemail_set}\pysigline{\code{Recipient.}\bfcode{myemail\_set}}
Accessor to the related objects manager on the forward and reverse sides of
a many-to-many relation.

In the example:

\begin{Verbatim}[commandchars=\\\{\}]
\PYG{k}{class} \PYG{n+nc}{Pizza}\PYG{p}{(}\PYG{n}{Model}\PYG{p}{)}\PYG{p}{:}
    \PYG{n}{toppings} \PYG{o}{=} \PYG{n}{ManyToManyField}\PYG{p}{(}\PYG{n}{Topping}\PYG{p}{,} \PYG{n}{related\PYGZus{}name}\PYG{o}{=}\PYG{l+s+s1}{\PYGZsq{}}\PYG{l+s+s1}{pizzas}\PYG{l+s+s1}{\PYGZsq{}}\PYG{p}{)}
\end{Verbatim}

\code{pizza.toppings} and \code{topping.pizzas} are \code{ManyToManyDescriptor}
instances.

Most of the implementation is delegated to a dynamically defined manager
class built by \code{create\_forward\_many\_to\_many\_manager()} defined below.

\end{fulllineitems}

\index{objects (push\_email.models.Recipient attribute)}

\begin{fulllineitems}
\phantomsection\label{push_email:push_email.models.Recipient.objects}\pysigline{\code{Recipient.}\bfcode{objects}\strong{ = \textless{}django.db.models.manager.Manager object\textgreater{}}}
\end{fulllineitems}


\end{fulllineitems}



\section{push\_email.tests module}
\label{push_email:module-push_email.tests}\label{push_email:push-email-tests-module}\index{push\_email.tests (module)}\index{EmailTestCase (class in push\_email.tests)}

\begin{fulllineitems}
\phantomsection\label{push_email:push_email.tests.EmailTestCase}\pysiglinewithargsret{\strong{class }\code{push\_email.tests.}\bfcode{EmailTestCase}}{\emph{methodName='runTest'}}{}
Bases: \code{django.test.testcases.TestCase}
\index{test\_email\_creation() (push\_email.tests.EmailTestCase method)}

\begin{fulllineitems}
\phantomsection\label{push_email:push_email.tests.EmailTestCase.test_email_creation}\pysiglinewithargsret{\bfcode{test\_email\_creation}}{}{}
\end{fulllineitems}


\end{fulllineitems}

\index{ThreadTestCase (class in push\_email.tests)}

\begin{fulllineitems}
\phantomsection\label{push_email:push_email.tests.ThreadTestCase}\pysiglinewithargsret{\strong{class }\code{push\_email.tests.}\bfcode{ThreadTestCase}}{\emph{methodName='runTest'}}{}
Bases: \code{django.test.testcases.TestCase}
\index{test\_thread() (push\_email.tests.ThreadTestCase method)}

\begin{fulllineitems}
\phantomsection\label{push_email:push_email.tests.ThreadTestCase.test_thread}\pysiglinewithargsret{\bfcode{test\_thread}}{}{}
\end{fulllineitems}


\end{fulllineitems}



\section{push\_email.utils module}
\label{push_email:module-push_email.utils}\label{push_email:push-email-utils-module}\index{push\_email.utils (module)}\index{EmailThread (class in push\_email.utils)}

\begin{fulllineitems}
\phantomsection\label{push_email:push_email.utils.EmailThread}\pysiglinewithargsret{\strong{class }\code{push\_email.utils.}\bfcode{EmailThread}}{\emph{username}, \emph{password}, \emph{imap\_id}}{}
Bases: \code{threading.Thread}
\index{check\_for\_new\_emails() (push\_email.utils.EmailThread method)}

\begin{fulllineitems}
\phantomsection\label{push_email:push_email.utils.EmailThread.check_for_new_emails}\pysiglinewithargsret{\bfcode{check\_for\_new\_emails}}{}{}
Checks for new emails that match the pattern ``{[}BBC423{]}{[}xx.x.xxxx{]} Agenda dd/mm/aaaa hh:mm''

\end{fulllineitems}

\index{run() (push\_email.utils.EmailThread method)}

\begin{fulllineitems}
\phantomsection\label{push_email:push_email.utils.EmailThread.run}\pysiglinewithargsret{\bfcode{run}}{}{}
Runs this thread

\end{fulllineitems}


\end{fulllineitems}

\index{add\_to\_calendar() (in module push\_email.utils)}

\begin{fulllineitems}
\phantomsection\label{push_email:push_email.utils.add_to_calendar}\pysiglinewithargsret{\code{push\_email.utils.}\bfcode{add\_to\_calendar}}{\emph{message}}{}
Formats an even to add it to google calendar
:param message: an email message
:return:

\end{fulllineitems}

\index{load\_already\_seen() (in module push\_email.utils)}

\begin{fulllineitems}
\phantomsection\label{push_email:push_email.utils.load_already_seen}\pysiglinewithargsret{\code{push\_email.utils.}\bfcode{load\_already\_seen}}{}{}
Loads messages that have already been seen in previous runs of this
code so that they don't appear as duplicate in the view for the user
:return:

\end{fulllineitems}

\index{login() (in module push\_email.utils)}

\begin{fulllineitems}
\phantomsection\label{push_email:push_email.utils.login}\pysiglinewithargsret{\code{push\_email.utils.}\bfcode{login}}{\emph{username}, \emph{password}, \emph{imap\_id}}{}
Tries to log in and return the imbox object
:param username:
:param password:
:param imap\_id:
:return: imbox

\end{fulllineitems}

\index{save\_email() (in module push\_email.utils)}

\begin{fulllineitems}
\phantomsection\label{push_email:push_email.utils.save_email}\pysiglinewithargsret{\code{push\_email.utils.}\bfcode{save\_email}}{\emph{message}}{}
Saves a given email in the database
:param message: email message
:return:

\end{fulllineitems}



\section{push\_email.views module}
\label{push_email:module-push_email.views}\label{push_email:push-email-views-module}\index{push\_email.views (module)}\index{EmailLoginView (class in push\_email.views)}

\begin{fulllineitems}
\phantomsection\label{push_email:push_email.views.EmailLoginView}\pysiglinewithargsret{\strong{class }\code{push\_email.views.}\bfcode{EmailLoginView}}{\emph{**kwargs}}{}
Bases: \code{django.views.generic.edit.FormView}

Renders the login view
\index{form\_class (push\_email.views.EmailLoginView attribute)}

\begin{fulllineitems}
\phantomsection\label{push_email:push_email.views.EmailLoginView.form_class}\pysigline{\bfcode{form\_class}}
alias of \code{LoginForm}

\end{fulllineitems}

\index{success\_url (push\_email.views.EmailLoginView attribute)}

\begin{fulllineitems}
\phantomsection\label{push_email:push_email.views.EmailLoginView.success_url}\pysigline{\bfcode{success\_url}\strong{ = `/home/'}}
\end{fulllineitems}

\index{template\_name (push\_email.views.EmailLoginView attribute)}

\begin{fulllineitems}
\phantomsection\label{push_email:push_email.views.EmailLoginView.template_name}\pysigline{\bfcode{template\_name}\strong{ = `index.html'}}
\end{fulllineitems}


\end{fulllineitems}

\index{EmailView (class in push\_email.views)}

\begin{fulllineitems}
\phantomsection\label{push_email:push_email.views.EmailView}\pysiglinewithargsret{\strong{class }\code{push\_email.views.}\bfcode{EmailView}}{\emph{**kwargs}}{}
Bases: \code{django.views.generic.base.View}

Renders the email view page
\index{post() (push\_email.views.EmailView method)}

\begin{fulllineitems}
\phantomsection\label{push_email:push_email.views.EmailView.post}\pysiglinewithargsret{\bfcode{post}}{\emph{request}}{}
Gets a POST request with the login data and sends it to the backend
:param request: POST request from the Login view
:return: None

\end{fulllineitems}

\index{template\_name (push\_email.views.EmailView attribute)}

\begin{fulllineitems}
\phantomsection\label{push_email:push_email.views.EmailView.template_name}\pysigline{\bfcode{template\_name}\strong{ = `emails.html'}}
\end{fulllineitems}


\end{fulllineitems}

\index{FullEmail (class in push\_email.views)}

\begin{fulllineitems}
\phantomsection\label{push_email:push_email.views.FullEmail}\pysiglinewithargsret{\strong{class }\code{push\_email.views.}\bfcode{FullEmail}}{\emph{email\_subject=None}, \emph{email\_from=None}, \emph{email\_body=None}, \emph{email\_date=None}, \emph{recipients=None}, \emph{copies=None}}{}
Bases: \code{object}

Creates a new Email object with all fields that will be used in the html templates

\end{fulllineitems}



\section{Module contents}
\label{push_email:module-contents}\label{push_email:module-push_email}\index{push\_email (module)}

\chapter{Models}
\label{modules/models:models}\label{modules/models::doc}

\chapter{Indices and tables}
\label{index:indices-and-tables}\begin{itemize}
\item {} 
\DUspan{xref,std,std-ref}{genindex}

\item {} 
\DUspan{xref,std,std-ref}{modindex}

\item {} 
\DUspan{xref,std,std-ref}{search}

\end{itemize}


\renewcommand{\indexname}{Python Module Index}
\begin{theindex}
\def\bigletter#1{{\Large\sffamily#1}\nopagebreak\vspace{1mm}}
\bigletter{p}
\item {\texttt{push\_email}}, \pageref{push_email:module-push_email}
\item {\texttt{push\_email.admin}}, \pageref{push_email:module-push_email.admin}
\item {\texttt{push\_email.apps}}, \pageref{push_email:module-push_email.apps}
\item {\texttt{push\_email.forms}}, \pageref{push_email:module-push_email.forms}
\item {\texttt{push\_email.migrations}}, \pageref{push_email.migrations:module-push_email.migrations}
\item {\texttt{push\_email.migrations.0001\_initial}}, \pageref{push_email.migrations:module-push_email.migrations.0001_initial}
\item {\texttt{push\_email.migrations.0002\_auto\_20160111\_2042}}, \pageref{push_email.migrations:module-push_email.migrations.0002_auto_20160111_2042}
\item {\texttt{push\_email.migrations.0003\_auto\_20160111\_2043}}, \pageref{push_email.migrations:module-push_email.migrations.0003_auto_20160111_2043}
\item {\texttt{push\_email.migrations.0004\_emailid}}, \pageref{push_email.migrations:module-push_email.migrations.0004_emailid}
\item {\texttt{push\_email.models}}, \pageref{push_email:module-push_email.models}
\item {\texttt{push\_email.tests}}, \pageref{push_email:module-push_email.tests}
\item {\texttt{push\_email.utils}}, \pageref{push_email:module-push_email.utils}
\item {\texttt{push\_email.views}}, \pageref{push_email:module-push_email.views}
\end{theindex}

\renewcommand{\indexname}{Index}
\printindex
\end{document}

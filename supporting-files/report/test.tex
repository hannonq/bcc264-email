\documentclass[10pt]{article}
\usepackage{scribe-md1}
\usepackage[usenames]{color}
\usepackage{multirow}
\usepackage[table]{xcolor}% http://ctan.org/pkg/xcolor
\usepackage{caption}


\Scribes{Hannon Queiroz, Marcos Vinícius}
\Lecturer{Carlos Frederico}
\LectureNumber{Email client - } %% vazio
\LectureDate{\today} % deixar vazio se não se refere a uma aula particular
\LectureTitle{Sistemas Operacionais} % ajustar para outros trabalhos
\Disciplina{BCC264}
\Periodo{2015/2}


% Default fixed font does not support bold face
\DeclareFixedFont{\ttb}{T1}{txtt}{bx}{n}{12} % for bold
\DeclareFixedFont{\ttm}{T1}{txtt}{m}{n}{12}  % for normal

% Custom colors
\usepackage{color}
\definecolor{deepblue}{rgb}{0,0,0.5}
\definecolor{deepred}{rgb}{0.6,0,0}
\definecolor{deepgreen}{rgb}{0,0.5,0}

\usepackage{listings}

% Python style for highlighting
\newcommand\pythonstyle{\lstset{
language=Python,
basicstyle=\ttm,
otherkeywords={self},             % Add keywords here
keywordstyle=\ttb\color{deepblue},
emph={MyClass,__init__},          % Custom highlighting
emphstyle=\ttb\color{deepred},    % Custom highlighting style
stringstyle=\color{deepgreen},
frame=tb,                         % Any extra options here
showstringspaces=false            % 
%postbreak=\raisebox{0ex}[0ex][0ex]{\ensuremath{\color{red}\hookrightarrow\space}}
}}


% Python environment
\lstnewenvironment{python}[1][]
{
\pythonstyle
\lstset{#1}
}
{}

% Python for external files
\newcommand\pythonexternal[2][]{{
\pythonstyle
\lstinputlisting[#1]{#2}}}

% Python for inline
\newcommand\pythoninline[1]{{\pythonstyle\lstinline!#1!}}


\begin{document}
\MakeScribeTop

Neste trabalho, foi implementado um sistema \textit{web} capaz de ler \textit{emails} de duas contas: uma do provedor \textit{gmail} e outra do \textit{yahoo}. 
Para realizar tal tarefa, o \textit{framework web Django} foi utilizado, a fim de reduzir o trabalho necessário para atender os requisitos de uma aplicação \textit{web}.\\

\section{Implementação}
Ambas as contas de \textit{email} são verificadas simultaneamente. Para isso, foi utilizada a biblioteca \textit{\textbf{threading}}, da biblioteca padrão da linguagem de programação \textit{Python}.

A \textit{thread} foi implementada da seguinte forma:

\begin{python}[breaklines=true]
class EmailThread(threading.Thread):
    def __init__(self, username, password, imap_id):
        """
        EmailThread constructor.
        :param username: username to log in the email account
        :param password: password for the given account
        :param imap_id: an ID to identify which imap code to use: 1 = gmail, 2 = yahoo
        """
        threading.Thread.__init__(self)

        self.username = username
        self.password = password
        self.imap_id = imap_id

    def run(self):
        """
        Runs this thread
        """

        print("Starting thread ", self.username)
        print()
        self.imbox = login(self.username, self.password, self.imap_id)

        load_already_seen()  # Loads messages that have already been seen in previous runs of this code so that they don't appear as duplicate in the view for the user
        self.check_for_new_emails()

    def check_for_new_emails(self):
        """
        Checks for new emails that match the pattern "[BBC423][xx.x.xxxx] Agenda dd/mm/aaaa hh:mm"
        """
        print("Checking for new emails on ", self.username)
        sleep_for = threading.Event()

        while True:
            unread_messages = self.imbox.messages()

            for uid, message in unread_messages:
                if  pattern.match(message.subject) and message.message_id not in already_seen:
                    EmailId.objects.create(
                        email_id=message.message_id
                    )
                    already_seen.add(message.message_id)  # workaround to avoid repeated emails

                    print("Subject: ", message.subject)
                    save_email(message)
                    add_to_calendar(message)

        print("%s is going to sleep for %d seconds"%(self.username, timeout))
        sleep_for.wait(timeout=timeout) # sleeps for n seconds
        print("%s is waking up"%(self.username))
\end{python}

Quando o usuário loga com as duas contas de \textit{email}, duas instâncias da \textit{thread} mostrada no código anterior são criadas.

\section{Condição de corrida}
Na implementação deste trabalho, não foram tratados casos onde condições de corrida possam ocorrer. O motivo para a não observação destes casos é que, apesar de teoricamente possíveis, condições de corrida são \textbf{extremamente} improváveis dadas as características da \textit{Internet}. Considere o seguinte cenário onde poderia haver condição de corrida:
\begin{enumerate}
 \item O email do Gmail recebe uma mensagem agendando um compromisso para o dia 30/01/2016 às 10:00.
 \item O email do Yahoo recebe este mesmo email, com a mesma data e hora, mas para um compromisso diferente.
 \item O cliente de email desenvolvido neste trabalho recupera os dois emails \textbf{ao mesmo tempo} e tenta adicionar os eventos no Google Calendar \textbf{ao mesmo tempo.}
\end{enumerate}

Mesmo que isso ocorresse, a própria latência da Internet faria com que cada requisição chegasse separadamente no servidor do Google Calendar, evitando que os dois eventos competissem entre si.



\end{document}
